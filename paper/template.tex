\documentclass[10pt]{article}

\usepackage{relsize}
\usepackage{wrapfig}
\usepackage{xstring}

%\usepackage[
%  paperwidth=5.5in,
%  paperheight=8.5in,
%  inner=1in, outer=0.4in,
%  top=0.8in,bottom=0.8in,
%]{geometry}

%\usepackage{amsfonts}
%\usepackage{amssymb}
%\usepackage{eucal}
%\usepackage{amsthm}


\usepackage{arev}
\usepackage[T1]{fontenc}


%\usepackage[adobe-utopia]{mathdesign}
%\usepackage{gfsartemisia-euler}

\usepackage{microtype}

\usepackage{tikz}
\usepackage{tikz-cd}
\usetikzlibrary{cd}

\newcommand{\myedge}{\rightskip = 0em plus 2em}
\def\myfoot#1{\footnote{\myedge#1}}

\def\mydef#1#2{\noindent\hangindent=1em{\bf #1}\hspace{0.5em}:\hspace{0.5em}#2

}

\setlength{\parskip}{0.5em}



\begin{document}

\title{Distributed Collaboration Protocol}
\author{Ken Stanton}
\date{\today}
\maketitle

\myedge

\setcounter{tocdepth}{1}
\tableofcontents
\newpage


\section{Introduction}
As Cardano grows, it will become increasingly difficult to know everything that's happening in Catalyst and the greater ecosystem.
The ways we interact with each other now will impact how things operate at scale.
It's important to look at how our current process would work with a billion people as well as improvements that can be made sooner to create a better user experience when that day comes.

Things are already getting to the point where it's impossible to know everything that's going on in Catalyst without an incredible amount of time and effort.
The optimal process for interacting with the community for the betterment of Cardano's future would embrace the fact that knowing everything is impossible.
The idea is to reduce the friction in people's pursuits of their goals.
What are people's motivations for interacting with Catalyst anyway?
It could be an interest in what projects are worth voting for.
Maybe, an interest in finding a team to collaborate with or a place to provide helpful feedback.
Finding interesting ideas for future work could also be a motivation.

Whatever motivation someone has to look through information on Catalyst, it stems from an interest in a topic or a skill-set that can be applied to future work.
Skill-sets are usually developed because of interests in the subjects they cover.
And, when a project isn't interesting to someone, it won't be given much time.
Gleaning from this insight, the best user experience would have to reduce the noise in the process of honing in on the topics that a person finds most interesting.
This entails allowing people to build up a fingerprint of their skills and interests, and matching that to the space of possible projects.

There is an important distinction to be made before progressing any further.
Finding a way to bypass topics that aren't interesting is not the same as allowing differing opinions on a specific topic to compete with each other in an individuals thoughts.
The disfunction found in social media algorithms seems to stem from a failure to make this distinction, and a lot of benefit could be achieved from a formal method of debating different opinions on any specific topic. 
A newcomer to a topic should be encouraged to look at it from many different points of view, as well as being allowed to provide insights from a personal perspective.

In order to reduce cognitive load, a process should be in place to converge on a minimal representation of an idea space without compromising its semantic structure.
This should also take into account different languages and different levels of fluency in a given language.
Convergence on a minimal grammar and vocabulary for expressing ideas would be helpful in crossing language barriers, and so would tools to study semantic overlap between terminology between different languages.
In language, abstract concepts are expressed in metaphoric constructions that map to more concrete semantic structures, like the way that time is expressed in terms of direction.
Providing a way to study differences in these mappings would be useful in the process of translation and the availability information to different groups.

Because no one will be able to track the entire system of Catalyst, and because different perspectives, interests, and languages don't always overlap, it doesn't even make sense to leave things under centralized control in the long run.
It could be beneficial to maintain some minimal structure that allows for cultural cohesion across different groups, but giving any individual a choice to pursue freedom and autonomy from the collective would be wise to consider as a basic human right.
This is one of the underlying motivations behind the rise of cryptocurrency---that banks and governments should not have control over an individual's financial freedom.
The same should be said about personal opinion.
Consolidation of power is a magnet for dictatorial tendencies, and the freedom to opt out of any given system is an important balancing mechanism for mitigating that problem. 

Over the course of this document, we will look at many different systems from many different fields and look for ways to describe the structures we find in simple terms.
We will also discuss the benefits and drawbacks of each system as it relates to producing optimal results within the constraints of the given system space.
Another topic to consider is the meta-theory behind the production of this document and how it could be improved by the community over time without infringing on my right to express my personal opinions.
Maintaining a historical context of idea development could play a role this non-infringement.
Ideas for the future of this document and how it might be improved through distributed collaboration should map quite well to the process of producing a proposal in the Catalyst project.
Having an adaptive and dynamic process for refining proposals should produce more robust results.
As far as voting mechanisms, funding, and the establishment of reputation goes, others are working on those problems so we won't focus on them here except if they naturally present themselves within the beneficial aspects of the systems we explore.
If my intuitions are correct, the way we view those things will be more productive if we take this approach.

Our next step in this process is to look at defining a common technical vocabulary to reduce the possibility of miscommunication on my part.
This can be seen as an instance of increasing cross-cultural availability of a set of ideas and how they relate to each other.

\section{Defining Vocabulary}
Here, we will be defining some terms in the most abstract way possible for the sake of giving them the most general scope of application.

%FIXME
\vspace{1.5em}\hrule\vspace{1em}
    \mydef{Culture}{Shared language, behavior, and/or motivations between two or more people.}
    \vspace{0.5em}
    \mydef{Language}{Patterns of thought propagated through some medium of exchange from one person to one or more others.}
    \vspace{0.5em}
    \mydef{Communication}{When language between different members of a group overlap enough for thoughts to be sufficiently conveyed.}
\vspace{1em}\hrule\vspace{1em}

%TODO: And more stuff!

\end{document}

